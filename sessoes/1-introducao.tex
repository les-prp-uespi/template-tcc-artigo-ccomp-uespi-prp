\section{Introdução}
\label{sec:intro}

A \textbf{introdução} do seu artigo deve apresentar as partes do seu artigo. De maneira geral, o ideal é que o leitor consiga entender \textit{(i)} o contexto, \textit{(ii)} o problema, \textit{(iii)} a literatura existente atacando aquele problema, \textit{(iv)} o método que você utilizou para enfrentar o problema, \textit{(v)} os resultados alcançados e \textit{(vi)} suas contribuições e \textit{(vii)} conclusões (ou considerações finais). Ela pode ser entendida como uma expansão do resumo do artigo. É comum ainda, que autores decidam por apresentar o restante do documento no decorrer do texto ou no seu último paragrafo como faremos a seguir.

O restante do documento é organizado como segue. {\color{blue} Todo texto \textbf{azul} que você encontrar em seguida é aleatório. Ele foi utilizado somente para preencher o documento.}
A Seção \ref{sec:fundamentacao} apresenta a fundamentação teórica. 
A Seção \ref{sec:metodologia} apresenta a metodologia
A Seção \ref{sec:resultados} apresenta os resultados (esperados no caso de projeto de pesquisa). 
Por fim, a Seção \ref{sec:conclusao} apresenta a conclusão do artigo.